% \usepackage[UTF8]{ctex}
\usepackage{mathptmx}
\usepackage{amsmath, amsthm, amssymb} %数学符号等相关宏包
\usepackage[utf8]{inputenc}
\usepackage{graphicx} %插入图片所需宏包
\usepackage[labelsep=space,labelfont=bf]{caption}
\usepackage{xspace} %提供一些好用的空格命令
\usepackage{url} %更好的超链接显示
\usepackage{float} %图片与表格的更好排版
\usepackage{ulem} %更好的下划线
\usepackage{setspace}% 行距等
\usepackage[a4paper,top=2.8cm, bottom=3.0cm, left=3.0cm, right=3.0cm]{geometry} %设置页边距
\usepackage{booktabs}
\usepackage{indentfirst}
\usepackage{titlesec}
\usepackage{fancyhdr}
\usepackage{makecell} %表格内换行
\usepackage{svg}
\usepackage{enumitem}
\usepackage{array}
\usepackage{abstract}
\usepackage{calc}
\usepackage{titletoc} % 设置目录格式
\usepackage{emptypage}
\usepackage{atbegshi}
\usepackage{blindtext}
\usepackage{times}
\usepackage{xcolor} %彩色的文字
\usepackage[colorlinks=true, linkcolor=black, citecolor=black, urlcolor=black,bookmarksnumbered=true]{hyperref}
\usepackage{cleveref} %交叉引用
\usepackage{listings}%code highlighting
\usepackage{bookmark}
%使用biblatex管理文献,输出格式使用gb7714-2015标准,后端为biber
\usepackage[backend=biber,style=gb7714-2015,hyperref=true]{biblatex}
\usepackage{pdfpages}

% 浮动体排版参数设置
\renewcommand{\floatpagefraction}{0.8} % 让页面至少 80% 是内容,避免图片单独占一页
\renewcommand{\textfraction}{0.1} % 让文字至少占 10% 页面
\renewcommand{\topfraction}{0.9} % 允许图片最多占 90% 页面
\renewcommand{\bottomfraction}{0.9} % 允许底部最多 90% 是图片

% 设置公式段前后间距
\everydisplay{\abovedisplayskip=15pt \belowdisplayskip=10pt}

% 列表项间距设置更紧凑
\setlist[enumerate]{itemsep=-1pt, topsep=0pt}
\setlist[itemize]{itemsep=-1pt, topsep=0pt}

% 避免行末长单词溢出页面
\sloppy

% 设置章节编号深度为 6
\setcounter{secnumdepth}{6}

% 设置正文为小4号字
\renewcommand{\normalsize}{\songti\zihao{-4}}

% 设置宋体为主字体,允许伪粗体与伪斜体
\setCJKfamilyfont{zhsong}[AutoFakeBold={2.5}, AutoFakeSlant=0.2]{SimSun}
\renewcommand*{\songti}{\CJKfamily{zhsong}}

% 指定中英文字体
\setCJKmainfont{SimSun}
\setmainfont{Times New Roman}

% 允许公式分页
\allowdisplaybreaks

% 设置 1.5 倍行距
\setstretch{1.5}

% 表格列间距设置
\setlength{\tabcolsep}{1mm}{}

% 图表编号按章节编号
\numberwithin{figure}{section}
\numberwithin{table}{section}

% 设置摘要标题与正文样式
\renewcommand{\abstractnamefont}{\heiti\zihao{4}}
\renewcommand{\abstracttextfont}{\zihao{-4}}
\setlength{\absleftindent}{0pt}
\setlength{\absrightindent}{0pt}


% 设置章节目录书签格式
\makeatletter
\bookmarksetup{%
    addtohook={%
    \ifnum\toclevel@section=\bookmarkget{level}\relax
        \renewcommand*{\numberline}[1]{第 #1 章 }%
    \fi
    },
}
\makeatother

% 设置章节标题格式
\titleformat
{\section}
{\centering\bfseries\songti\zihao{3}}
{第 \thesection 章}
{1em}
{}

\titleformat
{\subsection}
{\bfseries\songti\zihao{4}}
{\thesubsection}
{1em}
{}

\titleformat
{\subsubsection}
{\bfseries\songti\zihao{-4}}
{\thesubsubsection}
{1em}
{}

\titleformat
{\paragraph}
{\bfseries\songti\zihao{-4}}
{\theparagraph}
{1em}
{}

\titleformat
{\subparagraph}
{\bfseries\songti\zihao{-4}}
{\thesubparagraph}
{1em}
{}


\titlespacing{\subparagraph}{0pt}{1ex}{1em}  

% 自定义正文页码计数器
\newcounter{contentpage}
\setcounter{contentpage}{1}

% 每页开始前计数器+1
\AtBeginShipout{%
  \addtocounter{contentpage}{1}
}

% 页码对齐到奇数页(常用于章节起始页)
\newcommand{\alignOddPage}{
    % \cleardoublepage
    \ifodd\value{page}
        
    \else
        \thispagestyle{empty}
        \null
        \newpage
        \addtocounter{contentpage}{-1}
    \fi
}

   
% 设置页眉页脚样式
\pagestyle{fancy}
\fancyhf{} % 先清空样式
\fancyhead[L]{}
\fancyhead[R]{}
\fancyhead[C]{\vspace{0.4em}\zihao{-5}\kaishu\leftmark} % 页眉中间显示章节标题
\setlength{\headheight}{15.6404pt} % 页眉高度设置

\fancyfoot[R]{\bfseries ~\thecontentpage~} % 页脚右侧显示页码




% 定义罗马数字页码显示
\newcommand{\RomanNumbering}{
    \fancyfoot[R]{\bfseries ~\Roman{contentpage}~}
}
% 定义阿拉伯数字页码显示
\newcommand{\arabicNumbering}{
    \fancyfoot[R]{\bfseries ~\arabic{contentpage}~}
}
% 设置章节标题书签格式(用于页眉显示)
\renewcommand{\sectionmark}[1]{\markboth{第~\thesection~章~~#1}{}}

% 自定义三种对齐表格列类型(左、中、右)
\newcolumntype{L}[1]{>{\raggedright\let\newline\\\arraybackslash\hspace{0pt}}m{#1}}
\newcolumntype{C}[1]{>{\centering\let\newline\\\arraybackslash\hspace{0pt}}m{#1}}
\newcolumntype{R}[1]{>{\raggedleft\let\newline\\\arraybackslash\hspace{0pt}}m{#1}}

% 定义空白页命令
\newcommand{\emptypage}{
\null
\thispagestyle{empty}
\newpage
}

% 设置目录中各级标题格式
\titlecontents{section}
[0pt] % 左侧间距,增大这个值会加大subsection的缩进
{\addvspace{1.5pt}\filright\bf} % above code
{\contentspush{第\thecontentslabel\ 章\quad}} % numbered entry format
{} % numberless entry format
{\titlerule*[10pt]{.}\contentspage}  % filler page format

\titlecontents{subsection}
[1em] 
{\addvspace{0.5pt}} 
{\contentspush{\thecontentslabel\ }} 
{} 
{\titlerule*[10pt]{.}\contentspage} 

\titlecontents{subsubsection}
[2em] 
{\addvspace{0.5pt}} 
{\contentspush{\thecontentslabel\ }} 
{} 
{\titlerule*[10pt]{.}\contentspage} 